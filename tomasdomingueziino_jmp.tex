\documentclass[11pt]{article}
\usepackage{amssymb,amsmath,amsfonts,eurosym,geometry,ulem,graphicx,color,setspace,sectsty,comment,footmisc,caption,natbib,rotating,pdflscape,subfigure,array,float,dcolumn, bbm}
\usepackage[table,xcdraw,svgnames]{xcolor}
\usepackage{pstricks}
\usepackage[colorlinks]{hyperref}
\usepackage[en-US]{datetime2}
\usepackage[font=small]{caption}

%\usepackage{palatino}
%\DTMlangsetup{showdayofmonth=false}

%Colors
\definecolor{indigo(web)}{rgb}{0.29, 0.0, 0.51}
\definecolor{crimsonglory}{rgb}{0.75, 0.0, 0.2}
\definecolor{frenchblue}{rgb}{0.0, 0.45, 0.73}
\definecolor{deepcarrotorange}{rgb}{0.91, 0.41, 0.17}
\definecolor{dukeblue}{rgb}{0.0, 0.0, 0.61}
\definecolor{mediumblue}{rgb}{0.0, 0.0, 0.8}

\AtBeginDocument{
  \hypersetup{
    citecolor=indigo(web),
    linkcolor=indigo(web),   
    urlcolor=indigo(web)}}
\normalem
\onehalfspacing
\newtheorem{theorem}{Theorem}
\newtheorem{definition}{Definition}
\newtheorem{corollary}[theorem]{Corollary}
\newtheorem{proposition}{Proposition}
\newenvironment{proof}[1][Proof]{\noindent\textbf{#1.} }{\ \rule{0.5em}{0.5em}}
\newtheorem{hyp}{Hypothesis}
\newtheorem{subhyp}{Hypothesis}[hyp]
\renewcommand{\thesubhyp}{\thehyp\alph{subhyp}}
\newcolumntype{L}[1]{>{\raggedright\let\newline\\arraybackslash\hspace{0pt}}m{#1}}
\newcolumntype{C}[1]{>{\centering\let\newline\\arraybackslash\hspace{0pt}}m{#1}}
\newcolumntype{R}[1]{>{\raggedleft\let\newline\\arraybackslash\hspace{0pt}}m{#1}}
\geometry{left=1.0in,right=1.0in,top=1.0in,bottom=1.0in}

%%%%%%%%%%%%%%%%%%%%%%%%%%%%%%%%%%%%%%%%%%%%%%%%%%%%%%%%%%%%%%%%
\begin{document}

\begin{titlepage}


\title{Efficiency and Redistribution in Environmental Policy: \\
An Equilibrium Analysis of Agricultural Supply Chains}


\author{Tom\'{a}s Dom\'{i}nguez-Iino\thanks{\scriptsize PhD Candidate, Department of Economics, New York University. E-mail: \href{mailto:domingueziino@nyu.edu}{domingueziino@nyu.edu}. I thank Alessandro Lizzeri, Guillaume Fr\'echette, Elena Manresa, and Paul T. Scott for their guidance and support throughout this project. I also thank Milena Almagro, Tim Christensen, Christopher Flinn, Martin Rotemberg, Sharon Traiberman, Daniel Waldinger, seminar participants at NYU's Applied Micro Workshop, Econometrics Workshop, Stern Workshop, IO Reading Group, and conference participants at UPenn, Chicago, and Northwestern, for suggestions which have substantially improved this paper. Finally, I am indebted to multiple individuals at public and private research institutions in Argentina and Brazil for introducing me to institutional details, data availability, and the environmental science aspects of this project. Any errors or omissions are my own.
}
}

\date{\normalsize{
\vspace{0.15cm}
Job Market Paper\\
\today \\
\href{https://tdomingueziino.github.io/tomas_domingueziino_jmp.pdf}{Click here for latest version}
}
}

\maketitle

\begin{abstract}
\small{This paper provides an equilibrium framework to evaluate environmental policy in trade-exposed industries with imperfectly competitive supply chains. The empirical application is to the South American agricultural sector, a global agricultural powerhouse with a major environmental impact, whose trade flows are intermediated by a concentrated agribusiness sector. On the supply side, I innovate by introducing three key margins driving agricultural emissions---deforestation, commodity choice, and input substitution in livestock production. On the demand side, I innovate by introducing market power along the supply chain, requiring atomistic farmers to sell their output to monopsonistic intermediaries in order to access consumer markets. Given the infeasibility of a first-best carbon tax, I use my framework to evaluate second-best alternatives, such as environmental tariffs on imports from South America. Unless all trading partners regulate their imports, emissions reductions achieved by regulated markets are mostly offset by increased trade flows to non-regulated markets. Apart from being inefficient, unilateral tariffs have regressive distributional effects across space, since farmers in the poorest regions disproportionately bear the burden of environmental regulation through lower farm-gate prices. Agribusiness monopsony power exacerbates this effect due to higher pass-through rates onto farmers from these regions, since this is where supply is most inelastic. Thus, policies targeting a single externality can exacerbate other market distortions---not only in efficiency terms, but also in skewing the distribution of the remaining surplus.}
%\noindent\textbf{Keywords:} key1, key2, key3\\
%\vspace{0in}\\
%\noindent\textbf{JEL Codes:} key1, key2, key3\\
\end{abstract}
\setcounter{page}{0}
\thispagestyle{empty}
\end{titlepage}


\pagebreak \newpage
\doublespacing

\centering{
Full draft coming soon.\\
Please check \href{https://www.tomasdomingueziino.com}{www.tomasdomingueziino.com} for updates.
}


\end{document}